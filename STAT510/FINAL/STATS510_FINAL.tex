\documentclass[12pt,a4paper]{report}
\usepackage[utf8]{inputenc}
\usepackage[english]{babel}
\usepackage{enumerate}
\usepackage{amsmath}
\usepackage{amsfonts}
\usepackage{amssymb}
\usepackage{graphicx}
\usepackage{fourier}
\usepackage[left=2cm,right=2cm,top=2cm,bottom=2cm]{geometry}
\usepackage{commath}
\usepackage{cancel}
\usepackage{placeins}
%\author{Juan Carlos Apitz, ID 012523821}
%\title{Feature Selection Via Logistic Elastic Net Regression in Genetic Cancer Research}

\begin{document}

\begin{titlepage}
	\centering
	%\includegraphics[width=0.15\textwidth]{example-image-1x1}\par\vspace{1cm}
	{\scshape\Large Feature Selection Via Logistic Elastic Net Regression in Genetic Cancer Research \par}
	\vspace{1cm}
	{\scshape\large A PROJECT REPORT\par}
	%\vspace{1.5cm}
	{\large Presented to Department of Mathematics and Statistics California State University, Long Beach\par}
	\vspace{4cm}
	{\large In Partial Fulfillment of the Requirements for the Degree Master of Science in Mathematics Option in Statistics\par}
	\vspace{3cm}
	{\large Faculty Reviewer:\par}
	\vspace{0.5cm}
	{\large Kagba Suaray, Ph.D.}
	\vfill
	\large By ~Juan Carlos Apitz\\
	\large M.S., 2016, California State University, Long Beach\par

	\vfill

% Bottom of the page
	{\large \today\par}
\end{titlepage}





%\maketitle
%\begin{center}
%A Methodology Report\\
%

%\end{center}
%\clearpage
% Short Cuts%%%%%%%%%%%%%%%%%%%%%%%%%%%%%%%%%%%%%%%%%
\newcommand{\normalpdf}[1]{f \big( #1 \text{; } \mu,\sigma^2 \big) = \frac{1}{\sqrt{2 \pi \sigma^2}} e^{-\frac{1}{2 \sigma^2}(#1-\mu)^2}}

\newcommand{\expectedvalue}[1]{E \left(#1\right)}
%%%%%%%%%%%%%%%%%%%%%%%%%%%%%%%%%%%%%%%%%%%%%%%%%%%%%

\section*{Question 1}
The pdf of the random variable Y is:
\[
\normalpdf{y}
\]
Since Y is iid, we obtain the likelihood as follows:
\[
L\big(\mu\big) = \prod_{i=1}^n \left(2 \pi \sigma^2\right)^{-\frac{1}{2}} e^{-\frac{1}{2 \sigma^2}(y_i-\mu)^2}
\]

\[
= \left(2 \pi \sigma^2\right)^{-\frac{n}{2}} e^{-\frac{1}{2 \sigma^2}\sum_{i=1}^n(y_i-\mu)^2}
\]
To find the MLE of $\mu$, we proceed to maximize the $\log$ of $L\big(\mu\big)$ with respect to $\mu$. This is equivalent to maximizing $L\big(\mu\big)$ with respect to $\mu$. To accomplish this we can use the first and second derivatives tests from calculus to find the MLE:
\[
\ln{L\big(\mu\big)}=-\frac{n}{2}\ln{2 \pi \sigma^2}-\frac{1}{2 \sigma^2}\sum_{i=1}^n \big(y_i - \mu\big)^2
\]

\[
\frac{d \ln{L\big(\mu\big)}}{d\mu}=\frac{1}{\sigma^2}\left(\sum_{i=1}^n y_i - n \mu \right)\overset{set}{=}0
\]

\[
\hat{\mu}_{mle}=\frac{1}{n}\sum_{i=1}^n y_i
\]
The second derivative test confirms that $\hat{\mu}_{mle}$ is indeed a maximum:
\[
\frac{d^2 \ln{L\big(\mu\big)}}{d\mu^2}=-\frac{n}{\sigma^2}<0
\]
\clearpage
\section*{Question 2}
To find $\expectedvalue{\bar{Y}}$, we use the fact that the random variable $Y$ is distributed iid normal. Therefore we know that:

\[
\dfrac{\bar{Y}-\mu}{\frac{\sigma}{\sqrt{n}}} \sim N(0,1)
\]
Then:
\[
\expectedvalue{\dfrac{\bar{Y}-\mu}{\frac{\sigma}{\sqrt{n}}}}=0
\]
\[
\expectedvalue{\bar{Y}-\mu}=0
\]
\[
\expectedvalue{\bar{Y}}=\mu
\]
For the variance, we use the definition:
\[
V\left(\bar{Y}\right)=\expectedvalue{\bar{Y}-\expectedvalue{\bar{Y}}}^2
\]
and given that $Y$ is distributed iid normal we know:
\[
\expectedvalue{\dfrac{\bar{Y}-\mu}{\frac{\sigma}{\sqrt{n}}}-0}^2=1
\]
then:
\[
\expectedvalue{\bar{Y}-\mu}^2 = \dfrac{\sigma^2}{n}
\]

\[
\expectedvalue{\bar{Y}-\expectedvalue{\bar{Y}}}^2 = \dfrac{\sigma^2}{n}
\]

\[
V\left(\bar{Y}\right) = \dfrac{\sigma^2}{n}
\]
\clearpage
\section*{Question 3}
We can use linear algebra to solve the system of normal equations:
\[
\begin{pmatrix}
n & \sum_i x_i & \sum_i y_i\\
\\
\sum_i x_i & \sum_i x_i^2 & \sum_i x_i y_i
\end{pmatrix}
\]
Dividing the first row by $n$:
\[
\begin{pmatrix}
1 & \bar{x} & \bar{y}\\
\\
\sum_i X_i & \sum_i x_i^2 & \sum_i x_i y_i
\end{pmatrix}
\]
Multiply the first row by $\sum_i x_i$ and subtract it from the second row to get the second pivot:
\[
\begin{pmatrix}
1 & \bar{x} & \bar{y}\\
\\
0 & \sum_i x_i^2 - \bar{x}\sum_i x_i & \sum_i x_i y_i
 - \bar{y}\sum_i x_i
 \end{pmatrix}
\]
The above result implies that:
\[
b_1 = \dfrac{\sum_i x_i y_i
 - \bar{y}\sum_i x_i}{\sum_i x_i^2 - \bar{x}\sum_i x_i}
\]
To show that:
\[
b_1 = \dfrac{\sum_i\left(x_i-\bar{x}\right)\left(y_i - \bar{x}\right)}{\sum_i\left(x_i-\bar{x}\right)^2}
\]
expand $\sum_i\left(x_i-\bar{x}\right)\left(y_i - \bar{x}\right)$:
\[
= \sum_i\left(x_iy_i-\bar{y}x_i-\bar{x}y_i+\bar{x}\bar{y}\right)
\]

\[
= \sum_i x_i y_i-2\bar{y}\sum_i x_i+n\bar{x}\bar{y}
\]

\[
= \sum_i x_i y_i-2\bar{y}\sum_i x_i+\bar{y}\sum_i x_i
\]

\begin{equation} \label{eq:1}
= \sum_i x_i y_i-\bar{y}\sum_i x_i
\end{equation}
Now expand $\sum_i\left(x_i-\bar{x}\right)^2$:

\[
=\sum_i \left(x_i^2-2x_i\bar{x}+\bar{x}^2\right)
\]

\[
=\sum_i x_i^2-2\bar{x} \sum_i x_i+n \bar{x}^2
\]

\[
=\sum_i x_i^2-2\bar{x} \sum_i x_i+\bar{x}\sum_i x_i
\]

\begin{equation} \label{eq:2}
=\sum_i x_i^2-\bar{x} \sum_i x_i
\end{equation}
Based on result (\ref{eq:1}) and result (\ref{eq:2}), we have shown that:

\[
\dfrac{\sum_i\left(x_i-\bar{x}\right)\left(y_i - \bar{x}\right)}{\sum_i\left(x_i-\bar{x}\right)^2} = \dfrac{\sum_i x_i y_i
 - \bar{y}\sum_i x_i}{\sum_i x_i^2 - \bar{x}\sum_i x_i} = b_1
\]
\end{document}